\documentclass[11pt,a4paper,sans]{moderncv}
\moderncvstyle{classic} % style options are 'casual' (default), 'classic', 'oldstyle' and 'banking'
\moderncvcolor{blue} % color options 'blue' (default), 'orange', 'green', 'red', 'purple', 'grey' and 'black'
\usepackage{color}
\usepackage{fontspec}
\usepackage{fontawesome}
\usepackage[utf8]{inputenc} %% les accents dans le fichier.tex
\usepackage{csquotes}
\usepackage[\VAR{language}]{babel}
\usepackage[scale=0.78]{geometry} \usepackage{etoolbox}
\patchcmd{\section}{\vspace*{4ex}}{\vspace{4ex}}{}{}
\usepackage[autolang=other,backend=biber,dashed=false,sorting=none,maxbibnames=10,%
bibstyle=authoryear,url=false,doi=false,isbn=false]{biblatex}
\bibliography{publications}
\DefineBibliographyStrings{\VAR{language}}{pages = {pp\adddot}}
% Define \cvdoublecolumn, which sets its arguments in two columns without any labels

\newcommand{\cvtriplecolumn}[3]{%
  \cvitem[0.75em]{}{%
    \begin{minipage}[t]{0.3\linewidth}#1\end{minipage}%
    \hfill%
    \begin{minipage}[t]{0.3\linewidth}#2\end{minipage}%
    \hfill%
    \begin{minipage}[t]{0.3\linewidth}#3\end{minipage}%
    }%
}

\newcommand{\cvreference}[7]{%
    \textbf{#1}\newline% Name
    \ifthenelse{\equal{#2}{}}{}{\addresssymbol~#2\newline}%
    \ifthenelse{\equal{#3}{}}{}{#3\newline}%
    \ifthenelse{\equal{#4}{}}{}{#4\newline}%
    \ifthenelse{\equal{#5}{}}{}{#5\newline}%
    \ifthenelse{\equal{#6}{}}{}{\emailsymbol~\texttt{#6}\newline}%
    \ifthenelse{\equal{#7}{}}{}{\phonesymbol~#7}}

\nopagenumbers
\AtEveryBibitem{\clearfield{month}}
% personal data
\name{\VAR{first_name}}{\VAR{last_name}}
\title{\VAR{title}}
\email{\VAR{contact.mail}} % optional, remove / comment the line if not wanted
\homepage{\VAR{contact.homepage}}         % optional, remove / comment the line if not wanted
\social[linkedin]{\VAR{contact.linkedin}}      % optional, remove / comment the line if not wanted
\social[github]{\VAR{contact.github}}               % optional, remove / comment the line if not wanted
%\extrainfo{additional information}     % optional, remove / comment the line if not wanted
%\photo[64pt][0.4pt]{picture}          % optional, remove / comment the line if not wanted;
% '64pt' is the height the picture must be resized to, 0.4pt is the thickness of the frame
% around it (put it to 0pt for no frame) and 'picture' is the name of the picture file
%\quote{Some quote}

\begin{document}
\makecvtitle

\section{\VAR{experience.title}}
\BLOCK{ for x in experience.list }
\cventry{\VAR{x.period}}{\VAR{x.title}}{\VAR{x.company}}{\VAR{x.location}}{}{\VAR{x.description}}
\BLOCK{ endfor }

\BLOCK{ if teaching is defined }
\section{\VAR{teaching.title}}
\BLOCK{ for x in teaching.list }
\cventry{\VAR{x.period}}{\VAR{x.title}}{\VAR{x.description}}{}{}{}
\BLOCK{endfor}
\BLOCK{ endif }

\BLOCK{ if publications is defined }
\begin{refsection}
\BLOCK{ for x in publications.list }
\nocite{\VAR{x}}
\BLOCK{ endfor }
\printbibliography[title={\VAR{publications.title}}]
\end{refsection}
\BLOCK{ endif }

\BLOCK{ if conferences is defined }
\begin{refsection}
  \BLOCK{ for x in conferences }
  \nocite{\VAR{x.key}}
  \BLOCK{ endfor }
\printbibliography[title={\VAR{conferences.title}}]
\end{refsection}
\BLOCK{endif}

\BLOCK{ if peer_review is defined }
\section{\emph \VAR{peer_review.title}}
\BLOCK{ for x in peer_review }
\cvitemwithcomment{\VAR{x.title}}{}{\VAR{x.description}}
\BLOCK{ endfor }
\BLOCK{ endif }

\section{\VAR{education.title}}
\BLOCK{ for x in education.list }
\cventry{\VAR{x.period}}{\VAR{x.title}}{\VAR{x.company}}{\VAR{x.location}}{}{\VAR{x.description}} 
\BLOCK{ endfor }

\BLOCK{ if research_related is defined }
\section{\VAR{research_related.title}}
\BLOCK{ for x in research_related.list }
\cvitem{\VAR{x.title}}{\VAR{x.description}}
\BLOCK{ endfor }
\BLOCK{ endif }

\section{\VAR{skills.title}}
\BLOCK{ for x in skills.list }
\cvlistitem{\VAR{x.description}}
\BLOCK{ endfor }
\end{document}
